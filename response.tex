\documentclass{article}
\usepackage{units}
\begin{document}

We are greatful to the reviewer for their insightful suggestions to improve our manuscript.
Included below are the the reviewer's suggested changes, and our modifications to the manuscript.

\section{Savitsky-Golay filtering}

From the reviewer:

\begin{quote}
The authors also use a well-tested filtering method for accounting for interference.
I am curious, though, why was this method chosen over others? I suspect it may be that it is straightforward, which reduces the possibility of introducing more errors and biases, but it is not clear.
\end{quote}
\noindent We have added the following to our description of the filtering method used:

\begin{quote}
  This method was used because of the lack of edge effects by artificially extending the data, letting the filter be applied over the entire lightcurve.

\end{quote}



\section{Wind gust}

From the reviewer:

\begin{quote}
  Although a ``gust of wind'' is mentioned, it is not clearly indicated in the data when that gust occurred, so it would be helpful to point that out if the timing is known.
\end{quote}

\noindent To address this issue, we have modified the caption of Figure 1 to indicate the timing of the wind gust:

\begin{quote}
  The drop in the smoothed/normalized radio brightness due to the gust of wind is visible just after JD 2460409.22

\end{quote}

\noindent From the reviewer:

\begin{quote}
  Although something like that may not be possible to correct in the data analysis, it may be worth re-mentioning in the discussion where the radio light curve is compared to the theoretical curve.
\end{quote}

\noindent Because the wind gust occurred before the radio and optical lightcurves began to diverge, we have opted not to include further discussion of the wind gust in our comparison with theoretical lightcurves.

\section{Otsu thresholding}

From the reviewer:

\begin{quote}
  I would suggest, however, including a reference or very brief definition of Otsu thresholding.
\end{quote}

\noindent We have included the following reference: N. Otsu, “A threshold selection method from gray-level histograms,” IEEE Transactions on Systems, Man, and Cybernetics 9, 62–66 (1979)

\section{Model matching}

From the reviewer:

\begin{quote}
  It is not clear if the ``best fit curve'' was determined numerically or by eye
\end{quote}

\noindent The model was fit ``by eye.'' The following is not included in the theoretical lightcurve comparison section:

\begin{quote}
  We found our best-fitting solar raius by manually adjusting the model until the difference between the observed and model minima dropped below 1\%.

\end{quote}

\noindent From the reviewer:

\begin{quote}

  The variance in slope is particularly interesting! Is this also seen elsewhere in the literature?

\end{quote}

\noindent We have modified the conclusion as follows:

\begin{quote}
  In future work, we hope to examine the radio lightcurve in detail with the goal of analyzing the variance seen between our model slope and the recorded radio data. In a previous study by \cite{messerotti_radio_2000} the slopes also differ, but not to the degree seen in our data.
In \cite{messerotti_radio_2000}, that variance is used to map radio emission from the sun's surface as features are eclipsed by the moon's edge moving across the face of the sun, something we hope to replicate in a future study.

\end{quote}

\section{Comparison with other works}

From the reviewer:

\begin{quote}
  It is not clear why References 4 and 5 were chosen as the only reference points.
  I suspect it is because they were also measured at 1420 MHz and during a solar eclipse. If this is the case, it should be stated.
\end{quote}

\noindent We have included the following to the theoretical lightcurve comparison section:

\begin{quote}
  Both of these references observed partial solar eclipses at \unit[1420]{MHz}. In the case of \cite{leung_solar_2022}, those observations were made with the same radio receiver and software used in this study.

\end{quote}

\noindent From the reviewer:

\begin{quote}
  The authors should also consider comparing it to the long history of measurements of solar radii at radio wavelengths through other methods, such as those listed in Menezes \& Valio (2018).
  Does this measurement fit the general trend of radius with frequency?
\end{quote}

\noindent We have added the following to our comparision with other works:

\begin{quote}
  Our results are consistent with the trend found in \cite{menezes_solar_2017}, which compared decades of solar radio observations taken at many wavelengths, and found that solar radius increased as the radio frequency decreased (their Figure 5 and Equation 2).
This was also seen in the results of \cite{messerotti_radio_2000}, which observed the eclipse at four different wavelengths and found a similar trend (their Table 2.)




\end{quote}

\section{Emission source}

\noindent From the reviewer:

\begin{quote}
  Finally, indicate, perhaps in the introduction or discussion, what kind of emission IS being given off in the radio that gives us this larger radius? (It’s not from the blackbody spectrum that dominates in the visible.)
\end{quote}

\noindent We have added the following to the introduction, after mentioning that the telescope is tuned to the spin-flip transition frequency of neutral hydrogen:

\begin{quote}
  At this frequency, the radio emission from the sun is dominated by gyroresonance emission from electrons spiraling in the sun's extended magnetic field, which make the sun appear larger in radio frequencies than at optical frequencies.

\end{quote}
  



\bibliography{refs}% Produces the bibliography via BibTeX.
\bibliographystyle{plain}

\end{document}


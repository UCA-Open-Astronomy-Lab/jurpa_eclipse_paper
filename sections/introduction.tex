Our goal with this project is to determine the size of the sun in the radio frequency by using observations during the April 8th, 2024 total solar eclipse.
Our radio telescope is specifically focused on the electron spin-flip transition of Hydrogen, which is observed at 1420MHz.
We collected data in both the optical and radio frequencies, providing a unique opportunity to compare the relative size in the optical and radio spectra.
To accomplish this, we need to determine how much of the solar disk is obscured by the moon during totality in both the optical and radio wavelengths.
We have only one source of radio lightcurve data: UCA's 2.3 meter radio telescope.
In order to compare the radio eclipse to the optical eclipse, we have two sources of optical lightcurve data: a livestream from the UCA Observatory and a broad-spectrum light sensor pointed at the sky.
We also generated theoretical lightcurves using the software \texttt{Stellarium}\cite{zotti_simulated_2020} and the JPL Horizons On-Line Ephemeris System \cite{nasa_jpl_solar_system_dynamics_group_jpl_nodate}.
Using these data, we can compare the optical and radio lightcurves to determine the relative size of the radio solar disk.

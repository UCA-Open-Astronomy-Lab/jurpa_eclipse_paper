We present radio telescope observations of the 08 April 2024 total solar eclipse from the University of Central Arkansas campus in Conway, Arkansas using a \unit[2.3]{m} SPIDER 230C parabolic radio telescope tuned to a frequency of \unit[1420]{MHz}.
Observations began approximately \unit[19]{min} before first contact, and ended approximately \unit[2.5]{min} after fourth contact, continuously tracking the sun across the sky.
Our radio observations show a reduction in relative intensity from the beginning of the lightcurve to the middle of totality of approximately 70\%, indicating that the apparent size of the radio solar disk was larger than the apparent size of the moon and therefore only partially covered.
This contrasts with optical data, where the eclipse was total.
To determine the relative size of the radio solar disk, we compared our observed radio and optical data with theoretical curves where we varied the solar radius.
From our analyses, we found that the radio solar disk is approximately $R_{\mathrm{1420}} = 1.27 R_{\odot}$.
This is consistent with previously published results.
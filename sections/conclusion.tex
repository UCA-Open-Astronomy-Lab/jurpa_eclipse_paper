% Add Figure 5: Radio Lightcurve Data Analysis
In figure 5, we see a comparison between our normalized radio and optical lightcurves.
This comparision clearly shows a difference in brightness during totality, with the optical lightcurves reaching 0\% and the radio lightcurve reaching 30.9\%.
This indicates a difference in radius between the solar emissions from the optical and radio wavelengths.

% Add Figure 6: Radio Lightcurve Fitting
To analyze this difference in emission radius, we created theoretical lightcurves.
These  curves can be found in Figure 6, where we compare several theoretical solar radii with our collected radio data.
These curves, created using JPL Horizons On-Line Ephemeris System, were made such that they model a match of the relative difference in brightness from our radio observations.
That difference being 0\% in optical and 30.9\% in radio.
Using the angular radii of the sun and moon, along with their angular separation between each other, we calculated the obscured area of the solar disk at 2000 points in time during the eclipse using the following geometric relationships:
% Add Figure 7: Spheres
% Add Equations

These theoretical curves assume a uniform disk brightness, shown by the flat bottom of the lightcurve.
This flat bottom is not realistic, as limb darkening is not taken into account and the solar surface is non-uniform.
However, our result was consistent with our hypothesis: The radio solar emission comes from a larger radius than the optical.
The theoretical lightcurve that best matches the change in brightness of the radio eclipse is $R = 1.27R_s$.
Our result lies between the results of $R=1.14R_s$ found by Messerottie, et. al. (2000), and $R=1.4R_s$ found by Leung et. al. (2021).
In a future study, we will compare these and other results to examine potential sources of uncertainty in the measurements.
% Are the above future plans needed at all?
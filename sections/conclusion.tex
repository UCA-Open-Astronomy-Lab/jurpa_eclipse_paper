% Add Figure 5: Radio Lightcurve Data Analysis
In figure 5, we can see a comparison between our normalized radio and optical lightcurves.
This comparision clearly shows a difference in brightness during totality, with the optical lightcurves being at 0\% and the radio lightcurve being at 30.9\%.
%In that line, should it be "Being at 0 percent" or would it flow better to say "Reached 0 percent"?
This indicates that the radio solar emission comes from a larger radius than the optical.

% Add Figure 6: Radio Lightcurve Fitting
In Figure 6, we generated theoretical lightcurves for the 2024 eclipse.
% We need to site Stellarium
These theoretical curves assumed different solar radii to model a match of the relative drop in brightness from our radio observations.
In order to produce these lightcurves, we began with the high-accuracy ephemeris data from the JPL Horizons On-Line Ephemeris System.
Using the angular radii of both the sun and moon along with the angular separation between the two, we calculated the obscured area of the solar disk at 2000 points in time during the eclipse using the following geometric relationships:
% Add Figure 7: Spheres
% Add Equations

The theoretical lightcurve that best matches the depth of the radio eclipse is $R = 1.27R_s$.
This is consistent with our hypothesis of a larger radio disc than optical.
However, the overall profile of the lightcurve is not well fit by this result, and neither is the expected flat bottom if we assume uniform radio brightness across the surface of the sun.
Our result lies between the results of $R=1.14R_s$ found by Messerottie, et. al. (2000), and $R=1.4R_s$ found by Leung et. al. (2021).
In a future study we will compare these and other results to examine potential sources of uncertainty in the measurements.
% Are the above future plans needed at all?
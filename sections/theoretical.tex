\subsection{\label{sec:stellarium}Stellarium}

In order to check the validity of our livestream-derived optical lightcurve, we needed a way of finding the expected eclipse obscuration percentage over time for our location.
We used the free, open-source planetarium software \texttt{Stellarium} to achieve this.
After setting the time, date and location of our radio telescope in \texttt{Stellarium}, we queried the software's API to step the simulation time forward and extracted the eclipse percentage for 1000 points in time, starting before first contact and ending after fourth contact.

The resulting theoretical lightcurve is shown in Figure \ref{fig:livestream_stellarium_comparison}, alongside the eclipse obscruation derived from the UCA Observatory livestream.
The agreement between the two is excellent, validating our optical eclipse lightcurve.


\subsection{\label{sec:theoreticalLightcurves}Theoretical Lightcurve Comparison}


